%\documentclass[letterpaper,10pt,titlepage]{IEEEtran}
\documentclass[10pt, oneside,onecolumn,draftclsnofoot]{IEEEtran}

\usepackage{graphicx}
\usepackage{amssymb}
\usepackage{amsmath}
\usepackage{amsthm}

\usepackage{alltt}
\usepackage{float}
\usepackage{color}
\usepackage{url}

\usepackage{balance}
\usepackage[TABBOTCAP, tight]{subfigure}
\usepackage{enumitem}
\usepackage{pstricks, pst-node}

\usepackage{geometry}
\geometry{textheight=8.5in, textwidth=6in}

%random comment

\newcommand{\cred}[1]{{\color{red}#1}}
\newcommand{\cblue}[1]{{\color{blue}#1}}

\usepackage{hyperref}
\usepackage{geometry}
\usepackage{array}

\def\name{Jake Jeffreys, McKenna Jones, Spike Madden, Sean Marty}
\usepackage{titling}
\title{CS461: Client Requirements Document}
\author{Jake Jeffreys, McKenna Jones, Spike Madden, Sean Marty}
\date{October 25, 2016}

%pull in the necessary preamble matter for pygments output

%% The following metadata will show up in the PDF properties
\hypersetup{
  colorlinks = true,
  urlcolor = black,
  pdfauthor = {\name},
  pdfkeywords = {cs461 ``senior capstone''},
  pdftitle = {CS 461 Client Requirements Document},
  pdfsubject = {CS 461 Requirements Doc},
  pdfpagemode = UseNone
}

\begin{document}
\begin{titlepage}
\maketitle
\vspace{3cm}

\end{titlepage}

\tableofcontents
\section{Introduction}
todo
\subsection{Purpose}
The purpose of this project is to create a retail virtual reality experience for outdoor activities.

This document exists both for development of the project and for the clients
to provide a detailed description of the technical requirements.
\subsection{Definitions}
\begin{center}
	\begin{tabular}{| m{3cm} | m{9cm} |}
		\hline
		Virtual Reality & Sometimes shortened as VR \\
		\hline
		HTC Vive & A virtual reality headset produced by HTC \\
		\hline
		Unity Game Engine & The Unity Game Engine, developed by Unity Technologies is used in this project to develop the virtual reality simulation.
		\end{tabular}
\end{center}
\subsection{References}
todo

\section{Overall Description}
\subsection{Product Perspective}
This section is supposed to show how our project is linked to other related
projects. Many examples include a diagram but I don't know if that makes
sense for our project as it's pretty much a standalone project.
\subsection{Constraints}
A virtual reality headset like the HTC Vive has some inherent restrictions.
The first one is space. The HTC Vive tracks how much space you have set the
system up in and creates virtual barriers. This limitation can be alleviated by
using teleportation in the VR environment. A second limitation is the a
VR is not entirely realistic. Because we are trying to promote Columbia Sportswear
Gear it needs to be as realistic as possible, therefore this is a limitation.

\subsection{Product Functions}
The VR setup will allow the user to simiulate outdoor experiences.
Specifically, the user will be able to virtually tryon Columbia Sportswear
gear in the environments they are intended to be used in.
This will be done using the HTC Vive headset and wands. The final product
will allow the user to travel to a number of environments in a given session,
to test a wide range of gear.

Third or first person view?
Social aspect? Adding to cart?
\subsection{User Characteristics}
The general type of user of this project will be a customer at a Columbia
Sportswear retail store. Under this umbrella lies a few different types of
customers. First there are customers who are unexperienced in the outdoor
activity they are buying gear for. This target audience will benefit most
from the VR expreience as it will give them an idea of what they
are getting themselves into (too vague). Secondly there are customers who are
experienced in the activity they are buying gear for. This audience will
benefit from the VR experience because it will allow them to view themselves
actually using the gear.

\section{Specific Requirements}
todo
\subsection{External Interface Requirements}
todo
\subsection{Functional Requirements}
todo
\subsection{Performance Requirements}
todo
\subsection{Software System Attributes}
todo
\subsubsection{Reliability}
todo
\subsubsection{Availability}
todo
\subsubsection{Security}
todo
\subsubsection{Maintainability}
todo
\subsubsection{Portability}
todo

\vspace{3cm}

\noindent\begin{tabular}{ll}
\makebox[2.5in]{\hrulefill} & \makebox[2.5in]{\hrulefill}\\
Intel Sponsor & Date\\[8ex]% adds space between the two sets of signatures
\makebox[2.5in]{\hrulefill} & \makebox[2.5in]{\hrulefill}\\
Columbia Sponsor & Date\\[8ex]% adds space between the two sets of signatures
\makebox[2.5in]{\hrulefill} & \makebox[2.5in]{\hrulefill}\\[2ex]
\makebox[2.5in]{\hrulefill} & \makebox[2.5in]{\hrulefill}\\[2ex]
\makebox[2.5in]{\hrulefill} & \makebox[2.5in]{\hrulefill}\\[2ex]
\makebox[2.5in]{\hrulefill} & \makebox[2.5in]{\hrulefill}\\
Student Team Members & Date\\
\end{tabular}


\end{document}
