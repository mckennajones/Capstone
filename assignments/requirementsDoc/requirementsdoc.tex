%\documentclass[letterpaper,10pt,titlepage]{IEEEtran}
%\documentclass[10pt, oneside,onecolumn,draftclsnofoot]{IEEEtran}
\documentclass[10pt,journal,compsoc,onecolumn, draftclsnofoot]{IEEEtran}

\usepackage{graphicx}
\usepackage{amssymb}
\usepackage{amsmath}
\usepackage{amsthm}

\usepackage{alltt}
\usepackage{float}
\usepackage{color}
\usepackage{url}

\usepackage{balance}
\usepackage[TABBOTCAP, tight]{subfigure}
\usepackage{enumitem}
\usepackage{pstricks, pst-node}

\usepackage{geometry}
\usepackage{pst-gantt}

\geometry{textheight=8.5in, textwidth=6in}

%random comment

\newcommand{\cred}[1]{{\color{red}#1}}
\newcommand{\cblue}[1]{{\color{blue}#1}}

\usepackage{hyperref}
\usepackage{geometry}
\usepackage{array}
\usepackage{titling}

\def\name{Jake Jeffreys, McKenna Jones, Spike Madden, Sean Marty}
\title{
EmbarkVR: Outdoor Virtual Reality Experience \\
\vspace{1cm}
Software Requirements Specification \\
\vspace{3cm}
}
\author{Jake Jeffreys, McKenna Jones, Spike Madden, Sean Marty}
\date{November 4, 2016}

%pull in the necessary preamble matter for pygments output

%% The following metadata will show up in the PDF properties
\hypersetup{
  colorlinks = true,
  linkcolor = black,
  urlcolor = black,
  pdfauthor = {\name},
  pdfkeywords = {cs461 ``senior capstone''},
  pdftitle = {CS 461 Client Requirements Document},
  pdfsubject = {CS 461 Requirements Doc},
  pdfpagemode = UseNone
}

\begin{document}
\begin{titlepage}
\vspace{3cm}
\maketitle
\vspace{3cm}

\end{titlepage}

\tableofcontents
\clearpage
\section{Introduction}

\subsection{Purpose}
The main goal of the project is to make customers feel more inclined to purchase
Columbia gear through the use of an immersive, outdoor Virtual Reality
experience. This document exists both for development of the project and to provide a detailed description of the technical requirements for the clients.

\subsection{Scope}
We want to create an outdoor virtual reality experience for customers at a
Columbia retail store. The application will consist mainly of visual, audio, and tactile
experiences to create an outdoor world in which the user can navigate. The main
activity available will involve fly fishing in one of the rivers within the
environment. Users will also have the ability to interact with Columbia products
while in the experience and gain specific product information.

\subsection{Definitions}
\begin{center}
	\begin{tabular}{| m{3cm} | m{9cm} |}
		\hline
		Virtual Reality (VR) & Artificial environment that is created with software \\
		\hline
		HTC Vive & A virtual reality headset produced by HTC \\
		\hline
		Unity Game Engine & The Unity Game Engine, developed by Unity
		Technologies is used in this project to develop the virtual reality simulation. \\
    	\hline
    	Avatar & An icon or figure representing a particular person. \\
    	\hline
    	Wands & Controllers that are used with the HTC headset. \\
    	\hline
		Base Stations & These allow the Vive to track the movement and location of
    the wands and headset. \\
		\hline
	\end{tabular}
\end{center}

\subsection{References}
\begingroup
\renewcommand{\addcontentsline}[3]{}% Remove functionality of \addcontentsline
\renewcommand{\section}[2]{}% Remove functionality of \section
\bibliographystyle{IEEEtran}
\bibliography{requirementsdoc}
\endgroup

\subsection{Overview}
The next chapter of the document will give an overview of the basic functionality
of the Virtual Reality application. It contains informal requirements to provide
background for section three, Specific Requirements. Section three
will provide more detailed requirements and is intended for a more
technical audience such as developers.

\section{Overall Description}
\subsection{Product Perspective}
This VR product may be new to Columbia Sportswear but will still have ties to
existing products. Within the experience, users will have the ability to view
and interact with Columbia gear. Users will also be given the opportunity to
wear Columbia gear while participating to learn how the clothes feel while
executing certain movements.

The product will rely heavily on Virtual Reality Hardware. Specifically, the HTC
Vive System. This system consists of the the headset, two
wands, and two base stations. Additionaly a Virtual Reality compatible computer
is needed to actually run the software. In terms of software the
product will rely on the Unity Game Engine. Unity will do the heavy lifting
when it comes to rendering the virtual environment and making it look as realistic
as possible.

\subsection{Product Functions}
The VR setup will allow the user to simulate outdoor experiences.
Specifically, the user will be able to virtually see Columbia Sportswear
gear in the environments they are intended to be used in.
This will be done using 3D renderings of Columbia items which will then be placed
within the environment. We will then be using Unity to design VR interaction
capabilities on top of the renderings. The final product will allow the user to
test a variety of clothes and equipment. This product will also give customers
the ability to save the gear they liked in the VR, and access that information
after the experience is over.

\subsection{User Characteristics}
The general type of user of this project will be a customer at a Columbia
Sportswear retail store. Under this umbrella lies a few different types of
customers. First there are customers who are inexperienced in the outdoor
activity they are buying gear for. This target audience will benefit most
from the VR expreience as it will allow them to experience the activity without
a lot of economic or time commitment. Secondly, there are customers who are
experienced in the activity they are buying gear for. This audience will benefit
from the VR experience becauseit will allow them to view themselves actually
using and testing out new gear.

\subsection{Constraints}
A virtual reality headset like the HTC Vive has some inherent restrictions.
The first one is space. The HTC Vive tracks how much space you have set the
system up in and creates virtual barriers. This limitation can be alleviated by
using the controllers to move the users within VR environment. Besides physical
space, space in the virtual display is also a concern. Information needs to be
supplied to the user without obstructing the VR experience. A second
limitation are the graphics within the VR environemnt are not entirely
realistic. According to studies done by Intel and Thug Design \cite{michalak_lind_round1}\cite{michalak_lind_round2}, the categories most
important to the feeling of immersion are realistic interactions, responsiveness,
graphic clarity, and smooth transitions in that order. Because we are trying to
promote Columbia Sportswear Gear in a realistic environment it needs to be as
authentic as possible.

\subsection{Assumptions and Dependencies}
An important assumption made in this requirements document is that the virtual
reality experience will be run on a computer system that can run the HTC Vive
software.  The following are the minimum specifications to run Vive, as found
on the HTC Vive website \cite{htc_vive_ready}:
\begin{itemize}
  \item Processor: Intel Core i5-4590 or AMD FX 8350, or better
  \item Graphics: NVIDIA GeForce GTX 1060 or AMD Radeon RX480, or better
  \item Memory: 4GB RAM or better
  \item Operating system: Windows 7 SP1 or better
\end{itemize}
Design decisions and optimizations will be made so that a computer with the
above specifications can run the experience with little noticable lag, but if
the machine drops below the minimum capabilites, the requirements regarding
responsiveness will have to change.

Also, the requirements often depend on the availability of a set of Columbia
and Unity 3D assets.  If either of those sources of assets is not available,
the requirements about being able to see Columbia Sportswear gear in a
realistic environment will have to change.


\subsection{Apportioning of Requirements}
One part of the project that will likely be delayed until later versions is the
social aspect. Ideally the user would be able to share their VR experience on
social media sites like Facebook or Youtube. This could be in the form or 360
degree images or videos. At the moment this requirement is not a high priority.

\vspace{1cm}

Gant Chart (measured in weeks)
\begin{flushright}
\begin{PstGanttChart}[ChartShowIntervals,ChartUnitIntervalName=]{3}{12}
\PstGanttTask[TaskOutsideLabel={Build environment prototype}]{0}{4}
\PstGanttTask[TaskOutsideLabel={Build prototype of wand usage}]{4}{4}
\PstGanttTask[TaskOutsideLabel={Integrate wand usage with environment}]{8}{4}
\end{PstGanttChart}
\end{flushright}
\vspace{0.5cm}
\begin{flushright}
\begin{PstGanttChart}[ChartShowIntervals,ChartUnitIntervalName=,ChartStartInterval=13]{4}{14}
\PstGanttTask[TaskOutsideLabel={Add Columbia products with info}]{0}{4}
\PstGanttTask[TaskOutsideLabel={Add new user tutorial}]{4}{4}
\PstGanttTask[TaskOutsideLabel={Prepare for user Testing}]{8}{3}
\PstGanttTask[TaskOutsideLabel={Make changes and fix bugs}]{11}{3}
\end{PstGanttChart}
\end{flushright}

\section{Specific Requirements}
\subsection{External Interfaces}
\begin{itemize}
  \item 360 degree view of outdoor scenario within VR experience using HTC Vive
  headset. This will contain optional user guidance (visual) and offer Columbia
  product information (visual).
    \begin{itemize}
      \item Input: Movement of headset
      \item Output: Visual data
    \end{itemize}
  \item Immersive noises from outdoor VR experience. This includes audio from
  the optional user guidance.
    \begin{itemize}
      \item Output: Audio through speakers and/or headphones.
    \end{itemize}
  \item Ability for other users not using headset to see user's current view.
    \begin{itemize}
      \item Output: Visual data on external monitor.
    \end{itemize}
  \item Controller available to be held by user to interact within VR experience.
    \begin{itemize}
      \item Input: HTC Wand movement
    \end{itemize}
\end{itemize}

\subsection{Functions}
\begin{itemize}
  \item Ability for users to interact with fly fishing equipment.
  \item Ability to see Columbia fishing apparel in use.
\end{itemize}

\subsection{Performance Requirements}
\begin{itemize}
  \item Must maintain at least 60fps throughout experience.\cite{hall_2016}
\end{itemize}

\subsection{Software System Attributes}
Our unity environment should be portable and work on all HTC Vive systems.
Correctness can be evaluated by how authentic the real-world experience we're
trying to replicate is. We should be able to adjust the scenes easily and
accommodate for any changes that the client wants.

\subsubsection{Reliability}
The system will be considered reliable if it can provide the virtual reality
experience consistently without failure. Failure can be defined as any technical
issue that breaks immersion. This could include noticeable lag between the user
and avatar's actions, distortion of the environment, or a software defect that
causes the system to crash.

\subsubsection{Availability}
The system should be available to customers whenever the virtual reality station
in the store is set up. There should be no downtime when the customer puts on
the headset and the customer should almost immediately be placed in the immersive
environment.

Many of the attributes of our system (such as availability and reliability)
will be handled through the Unity game engine.

\subsubsection{Security}
The environment will be designed to not require a system with internet access,
and will not need any user information. The expected implementation will be on
a closed system in a Columbia store, where physical and remote access will be
controlled by Columbia employees.

\subsubsection{Maintainability}
As time goes on it should be easy to adapt the product to easily include more
Columbia Sportswear products and virtual reality environments to accompany them.
Changes to the environment through the addition of assets (new Columbia gear)
should not be difficult and an updated scene should be accomplished with a small patch.

\subsubsection{Portability}
The environment should work on all HTC Vive systems with the minimum
requirements to be Vive Ready \cite{htc_vive_ready}.

\clearpage

\section{Signatures}
Document: Software Requirements Specification (EmbarkVR)

\vspace{3cm}

\noindent\begin{tabular}{ll}
\makebox[2.5in]{\hrulefill} & \makebox[2.5in]{\hrulefill}\\
Intel Sponsor & Date\\[8ex]% adds space between the two sets of signatures
\makebox[2.5in]{\hrulefill} & \makebox[2.5in]{\hrulefill}\\
Columbia Sponsor & Date\\[8ex]% adds space between the two sets of signatures
\makebox[2.5in]{\hrulefill} & \makebox[2.5in]{\hrulefill}\\[2ex]
\makebox[2.5in]{\hrulefill} & \makebox[2.5in]{\hrulefill}\\[2ex]
\makebox[2.5in]{\hrulefill} & \makebox[2.5in]{\hrulefill}\\[2ex]
\makebox[2.5in]{\hrulefill} & \makebox[2.5in]{\hrulefill}\\
Student Team Members & Date\\
\end{tabular}

\end{document}
