%\documentclass[letterpaper,10pt,titlepage]{IEEEtran}
%\documentclass[10pt, oneside,onecolumn,draftclsnofoot]{IEEEtran}
\documentclass[10pt,journal,compsoc,onecolumn, draftclsnofoot]{IEEEtran}

\usepackage{graphicx}
\usepackage{amssymb}
\usepackage{amsmath}
\usepackage{amsthm}

\usepackage{alltt}
\usepackage{float}
\usepackage{color}
\usepackage{url}

\usepackage{balance}
\usepackage[TABBOTCAP, tight]{subfigure}
\usepackage{enumitem}
\usepackage{pstricks, pst-node}

\usepackage{geometry}
\usepackage{pst-gantt}
\usepackage{tabu}

\geometry{textheight=8.5in, textwidth=6in}

%random comment

\newcommand{\cred}[1]{{\color{red}#1}}
\newcommand{\cblue}[1]{{\color{blue}#1}}

\usepackage{hyperref}
\usepackage{geometry}
\usepackage{array}
\usepackage{titling}

\def\name{Jake Jeffreys, McKenna Jones, Spike Madden, Sean Marty}
\title{
EmbarkVR: Outdoor Virtual Reality Experience \\
CS Senior Capstone \\
Design Document\\
\vspace{3cm}
}
\author{Jake Jeffreys, McKenna Jones, Spike Madden, Sean Marty}
\date{November 17th, 2016}

%pull in the necessary preamble matter for pygments output

%% The following metadata will show up in the PDF properties
\hypersetup{
  colorlinks = true,
  linkcolor = black,
  urlcolor = black,
  pdfauthor = {\name},
  pdfkeywords = {cs461 ``senior capstone''},
  pdftitle = {CS 461 Design Document},
  pdfsubject = {CS 461 Design Document},
  pdfpagemode = UseNone
}

\begin{document}
\begin{titlepage}
\vspace{3cm}
\maketitle
\vspace{3cm}
\begin{abstract}
Abstract Goes Here
\end{abstract}

\end{titlepage}
\tableofcontents
\clearpage

\section{Overview}
\subsection{Scope}
We want to create an outdoor virtual reality experience for customers at a Columbia retail store. The application will consist mainly of visual, audio, and tactile experiences to create an outdoor world in which the user can navigate. The main activity available will involve fly fishing in one of the rivers within the environment. Users will also have the ability to interact with virtual Columbia products while in the experience and gain specific product information.

\subsection{Purpose}
The main goal of the project is to make customers feel more inclined to purchase Columbia gear through the use of an immersive, outdoor Virtual Reality experience. This document exists both for development of the project and to provide a detailed description of the design plans.

\subsection{Intended Audience}
The intended audience of this design document are the student developers involved (EmbarkVR), project sponsors, and Capstone teachers. The development team will be using this report as a guide and will provide structure for the development process. The sponsors can use this document to understand the vision of the developers and to will give a platform to discuss design ideas. The teachers can benefit from this document by learning about the project as a whole.

\section{Definitions}
\begin{itemize}
  \item Virtual Reality (VR):  Artificial environment that is created with software
  \item HTC Vive: A virtual reality headset produced by HTC
  \item Base Stations: These allow the Vive to track the movement and location of the wands and headset.
  \item Wands: Controllers that are used with the HTC headset.
  \item Unity Game Engine: The Unity Game Engine, developed by Unity Technologies is used in this project to develop the virtual reality simulation.
  \item GitHub: Web-based Git repository hosting service
  \item Git: version control system used for software development
\end{itemize}

\section{Project Context}
\subsection{Hardware}
\begin{itemize}
  \item Laptop Computers with the following specifications:
  \begin{itemize}
    \item Processor: Intel Core i5-4590 or AMD FX 8350, or better
    \item Graphics: NVIDIA GeForce GTX 1060 or AMD Radeon RX480, or better
    \item Memory: 4GB RAM or better
    \item Operating system: Windows 7 SP1 or better
  \end{itemize}
  \item HTC Vive Headset: Used to track head movements and display application to users.
  \item HTC Wands (x2): Used to track the users hand movements and to give the user the ability to interact with virtual objects within the application.
  \item HTC Base Stations (x2): Used to track location of headset and wands. This is information is then sent back to the computer in real time.
\end{itemize}
\subsection{Software}
\begin{itemize}
  \item Unity Gaming Engine: Used to develop the application.
  \item Unity Asset Store: Used to find objects which can be imported into the application.
  \item GitHub: Used by developers to collaborate and share files.
\end{itemize}


\section{Design Description}
\subsection{SDD identification}
\subsection{Design stakeholders}
\subsubsection{Intel}
Intel is working with Columbia Sportwear to help them meet their needs when it comes to this Outdoor Simulation Project.
Intel has graciously provided all necessary hardware to our team to allow us to create a successful application.

\subsubsection{Columbia Sportswear}
One aspect of Columbia Sportwear is their fishing apparel.
Specifically, the Performace Fishing Gear (PFG) line of apparel.
Columbia hopes to use the application we are developing in a retail store to showcase the PFG line in a new medium.
The goal is inspire customers to try new outdoor activities with Columbia gear.

\subsection{Design views}
\subsubsection{Users}
Users of the product expect this virtual reality experience to be as realistic and immersive as possible.
There are two main perspectives that users will have when using this product.
Firstly, users will be hoping to gain an outdoor experience that they might not otherwise have the opportunity to try.
Therefore, realism is key in this view.
Secondly, users will expect interaction with Columbia gear in a meaningful way.
The user should leave the experience with a feeling of how the Columbia gear would perform in a certain environment.

\subsubsection{Intel Sponsor (Mike Premi)}
The Intel sponsor of the project, Mike Premi, is concerned more with the technical side of the project.
Things like which techonologies are used, the techical performace, and overall technical design considerations are important under this view.
This view will guide the design process on a technical level.

\subsubsection{Columbia Sponsor (Tim Devlin)}
The Columbia Sportwear sponsor of the project, Tim Devlin, is concerened primarily with the how the user will interact with Columbia products in the Virtual Reality experience.
This includes, how products are displayed, what information related to the products is shown, and the user interaction with said products.
Ultimately, the goal of the product under this view is to create more sales for Columbia Sportswear.
Therefore, that is what is most important under this view.
This view will gude the design process of a higher level compared to the Intel Sponsor view.

\subsection{Design viewpoints}
\subsubsection{Context viewpoint}
*[Context viewpoint depicts services provided by a design subject with reference to an explicit context. That context is defined by reference to actors that include users and other stakeholders, which interact with the design subject in its environment. The Context viewpoint provides a “black box” perspective on the design subject.]*
\begin{itemize}
  \item Design Concern:
  \item Analytical Methods
  \item Rationale
\end{itemize}

\subsubsection{Composition viewpoint}
*[Composition viewpoint describes the way the design subject is (recursively) structured into constituent parts and establishes the roles of those parts.]*
\begin{itemize}
  \item Design Concern:
  \item Analytical Methods
  \item Rationale
\end{itemize}

\subsubsection{Dependency viewpoint}
*[The Dependency viewpoint specifies the relationships of interconnection and access among entities. These relationships include shared information, order of execution, or parameterization of interfaces.]*
\begin{itemize}
  \item Design Concern:
  \item Analytical Methods
  \item Rationale
\end{itemize}

\subsubsection{Interface viewpoint}
*[Interface viewpoint provides information designers, programmers, and testers the means to know how to correctly use the services provided by a design subject. This description includes the details of external and internal interfaces not provided in the SRS.]*
\begin{itemize}
  \item Design Concern:
  \item Analytical Methods
  \item Rationale
\end{itemize}


\section{Approach}
\subsection{Static Environment}
(terrain, static objects, Columbia gear assets)

\subsection{Animated Environment}
One of the main aspects of our project is to make it as realistic as possible without compromising performance. Realism can come from a number of different techniques. The first we will be focusing on is environment animation. A majority of our application will take place in a river so we will need to make this river as animated as possible. We will need to add an animation of the water moving passed the users as well as associated audio. Audio is crucial when it comes to immersion so not only will we need to add water noises but also noises related to wind and a wide range of animals. The last technique we will be focusing to improve realism is lighting and shadowing.

\subsection{Tactile User Interaction}
(user interaction with gear, Columbia gear info)

\subsection{Rod mechanics}
In order to create a realistic fishing experience, the user will need to be able to interact with a virtual fishing rod.
The user's interaction with the rod will be primarily based around the use of the HTC Vive controllers.
Like other virtual reality simulations, in the game you will not see the Vive controllers, but instead virtual hands.
The user will then be able to pick up the fishing rod using these virtual hands.
To make this interaction as natural as possible the VR hands need to feel like an actual extension of the user's body.
Once the user has picked up the fishing rod, it needs to behave as an actual rod would.
This means that we will be using Unity's 3D physics engine extensively to create realistic movements with the fishing rod, line, and bait.

% \bibliographystyle{IEEEtran}
% \bibliography{designdoc}
\end{document}
