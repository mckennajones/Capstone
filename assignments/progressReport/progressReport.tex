%\documentclass[letterpaper,10pt,titlepage]{IEEEtran}
%\documentclass[10pt, oneside,onecolumn,draftclsnofoot]{IEEEtran}
\documentclass[10pt,journal,compsoc,onecolumn, draftclsnofoot]{IEEEtran}

\usepackage{graphicx}
\usepackage{amssymb}
\usepackage{amsmath}
\usepackage{amsthm}
\usepackage{caption}

\usepackage{alltt}
\usepackage{float}
\usepackage{color}
\usepackage{url}

\usepackage{balance}
\usepackage[TABBOTCAP, tight]{subfigure}
\usepackage{enumitem}
\usepackage{pstricks, pst-node}

\usepackage{geometry}
\usepackage{pst-gantt}
\usepackage{tabu}

\geometry{textheight=8.5in, textwidth=6in}

%random comment

\newcommand{\cred}[1]{{\color{red}#1}}
\newcommand{\cblue}[1]{{\color{blue}#1}}

\graphicspath{ {diagrams/} }

\usepackage{hyperref}
\usepackage{geometry}
\usepackage{array}
\usepackage{titling}

\def\name{Jake Jeffreys, McKenna Jones, Spike Madden, Sean Marty}
\title{
EmbarkVR: Outdoor Virtual Reality Experience \\
CS Senior Capstone \\
Progress Report\\
\vspace{1mm}
}
\author{Jake Jeffreys, McKenna Jones, Spike Madden, Sean Marty}
\date{4 December 2016}

%pull in the necessary preamble matter for pygments output

%% The following metadata will show up in the PDF properties
\hypersetup{
  colorlinks = true,
  linkcolor = black,
  urlcolor = black,
  pdfauthor = {\name},
  pdfkeywords = {cs461 ``senior capstone''},
  pdftitle = {CS 461 Progress Report},
  pdfsubject = {CS 461 Progress Report},
  pdfpagemode = UseNone
}

\begin{document}
\begin{titlepage}
\maketitle
\vspace{1mm}
\begin{abstract}
ABSTRACT
\end{abstract}
\vspace{1cm}

\noindent\begin{tabular}{ll}
\makebox[2.5in]{\hrulefill} & \makebox[2.5in]{\hrulefill}\\
Intel Sponsor & Date\\[5ex]% adds space between the two sets of signatures
\makebox[2.5in]{\hrulefill} & \makebox[2.5in]{\hrulefill}\\
Columbia Sponsor & Date\\[5ex]% adds space between the two sets of signatures
\makebox[2.5in]{\hrulefill} & \makebox[2.5in]{\hrulefill}\\[2ex]
\makebox[2.5in]{\hrulefill} & \makebox[2.5in]{\hrulefill}\\[2ex]
\makebox[2.5in]{\hrulefill} & \makebox[2.5in]{\hrulefill}\\[2ex]
\makebox[2.5in]{\hrulefill} & \makebox[2.5in]{\hrulefill}\\
Student Team Members & Date\\
\end{tabular}

\end{titlepage}
\tableofcontents
\clearpage

\section{Project Overview}
This project aims to create a functional and immersive virtual reality outdoor experience that promotes Columbia gear to outdoor enthusiasts and newcomers.
Many outdoor activities initially require a large mental and economic
investment to get started.
This makes people less likely to try new outdoor activities.
The goal of the project is to develop an interactive product demonstration to combat this issue.
This will be accomplished through the use of the HTC Vive and the Unity Game Engine.

\section{Problems Encountered}
Throughout the Fall 2016 term of Capstone, our group encountered several problems.
\subsection{Scheduling}
Scheduling development time and meeting times was, at first, a struggle.
We found it hard to manage the schedules of four group members, two sponsors, and a TA.
We were able to sort it out and meet regularly to work on the assigned documents and our VR prototype.
\subsection{Unity Requirements}
Two of our group members had some trouble with running Unity on their laptops.
Fortunately, our sponsors were able to send us our own Vive and laptop for development purposes.
\subsection{Documentation}
As this term's work mostly focused on documentation, we ran into several problems across the various documents.
The different structures of the documents was a reoccurring issue.
A lot of the requirements wern't clear and the IEEE format contradicted the suggested format from the assignment sheet and professors.

\section{Weekly Progress}
\subsection{Week 1}
\subsection{Week 2}
\subsection{Week 3}
\subsection{Week 4}
\subsection{Week 5}
\subsection{Week 6}
\subsection{Week 7}
\subsection{Week 8}
\subsection{Week 9}
\subsection{Week 10}

\section{Retrospective}

\section{Current Status}







%\clearpage
%\bibliographystyle{IEEEtran}
%\bibliography{designdoc}
\end{document}
