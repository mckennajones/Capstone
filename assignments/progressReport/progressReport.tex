%\documentclass[letterpaper,10pt,titlepage]{IEEEtran}
%\documentclass[10pt, oneside,onecolumn,draftclsnofoot]{IEEEtran}
\documentclass[10pt,journal,compsoc,onecolumn, draftclsnofoot]{IEEEtran}

\usepackage{graphicx}
\usepackage{amssymb}
\usepackage{amsmath}
\usepackage{amsthm}
\usepackage{caption}

\usepackage{alltt}
\usepackage{float}
\usepackage{color}
\usepackage{url}

\usepackage{balance}
\usepackage[TABBOTCAP, tight]{subfigure}
\usepackage{enumitem}
\usepackage{pstricks, pst-node}

\usepackage{geometry}
\usepackage{pst-gantt}
\usepackage{tabu}

\geometry{textheight=8.5in, textwidth=6in}

%random comment

\newcommand{\cred}[1]{{\color{red}#1}}
\newcommand{\cblue}[1]{{\color{blue}#1}}

\graphicspath{ {diagrams/} }

\usepackage{hyperref}
\usepackage{geometry}
\usepackage{array}
\usepackage{titling}

\def\name{Jake Jeffreys, McKenna Jones, Spike Madden, Sean Marty}
\title{
EmbarkVR: Outdoor Virtual Reality Experience \\
CS Senior Capstone \\
Progress Report\\
\vspace{1mm}
}
\author{Jake Jeffreys, McKenna Jones, Spike Madden, Sean Marty}
\date{4 December 2016}

%pull in the necessary preamble matter for pygments output

%% The following metadata will show up in the PDF properties
\hypersetup{
  colorlinks = true,
  linkcolor = black,
  urlcolor = black,
  pdfauthor = {\name},
  pdfkeywords = {cs461 ``senior capstone''},
  pdftitle = {CS 461 Progress Report},
  pdfsubject = {CS 461 Progress Report},
  pdfpagemode = UseNone
}

\begin{document}
\begin{titlepage}
\maketitle
\vspace{1mm}
\begin{abstract}
ABSTRACT
\end{abstract}
\vspace{1cm}
\end{titlepage}
\tableofcontents
\clearpage

\section{Project Overview}

\section{Problems Encountered}

\section{Weekly Progress}
\subsection{Week 1}
\subsection{Week 2}
\subsection{Week 3}
\subsection{Week 4}
\subsection{Week 5}
\subsection{Week 6}
\subsection{Week 7}
\subsection{Week 8}
\subsection{Week 9}
\subsection{Week 10}

\section{Retrospective}
The following table represents a reflection of the previous development cycle.
It describes the positives, changes that need to be implimented, and actions required to implimennt these changes.

\begin{center}
\begin{tabular}{ |c || c | c | }
 \hline
 Positives & Deltas & Actions \\
 \hline \hline
 Collaboration with other & Need more development time & Schedule weekly team work sessions\\
 Intel VR team & & \\ \hline
 Good relationship with & More detail in project & -Learn from experience during \\ project sponsors & documentation & development \\
 & & -Submit documentation updates to \\
 & & professors and project sponsors\\ \hline
 Completed documentation & Need to obtain apparel & Setup meeting with Tim to discuss\\
 & assets from Columbia & \\ \hline
 Fully functional hardware setup & Decide on performace metrics & Further research on what\\
 & & makes VR experience successful\\ \hline
 Developed "Hello World" & & \\
 Unity program & & \\
 \hline
\end{tabular}
\end{center}
\section{Current Status}
As discussed above, the past term has primarily been devoted to developing documentation.
We spent the term carefully planning out every small aspect of our project.
While this was a lot of work, and not easy by any means, it will make the development process infinitely easier.
As we develop the project, the documentation is bound to change, but it will still serve as a blueprint for development.

Other than writing documentation, we also began learning Unity as none of our team had previous experience with the gaming engine.
At this point in time we have successfully completed a "Hello World" type program in Unity and displayed it on the Vive headset.
This was a big step for our team.
From here we have began the initial work on a functional prototype.
Currently it is simply a basic river and surrounding landscape that was constructed from some of the default Unity assets.
With the limited amount of free time we have found, this is a great start.
After the term ends our team will be staying in Corvallis for a couple of days to work on the prototype.
The goal is to have a basic functional VR fishing prototype by the beginning of next term.
This is a large undertaking, but our team is motivated to get flush out the prototype.
It will set us up for success going into next term.

%\clearpage
%\bibliographystyle{IEEEtran}
%\bibliography{designdoc}
\end{document}
