%\documentclass[letterpaper,10pt,titlepage]{IEEEtran}
%\documentclass[10pt, oneside,onecolumn,draftclsnofoot]{IEEEtran}
\documentclass[10pt,journal,compsoc,onecolumn, draftclsnofoot]{IEEEtran}

\usepackage{graphicx}
\usepackage{amssymb}
\usepackage{amsmath}
\usepackage{amsthm}
\usepackage{caption}

\usepackage{alltt}
\usepackage{float}
\usepackage{color}
\usepackage{url}

\usepackage{balance}
\usepackage[TABBOTCAP, tight]{subfigure}
\usepackage{enumitem}
\usepackage{pstricks, pst-node}

\usepackage{geometry}
\usepackage{pst-gantt}
\usepackage{tabu}

\geometry{textheight=8.5in, textwidth=6in}

%random comment

\newcommand{\cred}[1]{{\color{red}#1}}
\newcommand{\cblue}[1]{{\color{blue}#1}}

\graphicspath{ {diagrams/} }

\usepackage{hyperref}
\usepackage{geometry}
\usepackage{array}
\usepackage{titling}

\def\name{Jake Jeffreys, McKenna Jones, Spike Madden, Sean Marty}
\title{
EmbarkVR: Outdoor Virtual Reality Experience \\
CS Senior Capstone \\
Progress Report\\
\vspace{1mm}
}
\author{Jake Jeffreys, McKenna Jones, Spike Madden, Sean Marty}
\date{4 December 2016}

%pull in the necessary preamble matter for pygments output

%% The following metadata will show up in the PDF properties
\hypersetup{
  colorlinks = true,
  linkcolor = black,
  urlcolor = black,
  pdfauthor = {\name},
  pdfkeywords = {cs461 ``senior capstone''},
  pdftitle = {CS 461 Progress Report},
  pdfsubject = {CS 461 Progress Report},
  pdfpagemode = UseNone
}

\begin{document}
\begin{titlepage}
\maketitle
\vspace{1mm}
\begin{abstract}
ABSTRACT
\end{abstract}
\vspace{1cm}
\end{titlepage}
\tableofcontents
\clearpage

\section{Project Overview}
This project aims to create a functional and immersive virtual reality outdoor experience that promotes Columbia gear to outdoor enthusiasts and newcomers.
Many outdoor activities initially require a large mental and economic
investment to get started.
This makes people less likely to try new outdoor activities.
The goal of the project is to develop an interactive product demonstration to combat this issue.
This will be accomplished through the use of the HTC Vive and the Unity Game Engine.


\section{Problems Encountered}
Throughout the Fall 2016 term of Capstone, our group encountered several problems.
\subsection{Scheduling}
Scheduling development time and meeting times was, at first, a struggle.
We found it hard to manage the schedules of four group members, two sponsors, and a TA.
We were able to sort it out and meet regularly to work on the assigned documents and our VR prototype.
\subsection{Unity Requirements}
Two of our group members had some trouble with running Unity on their laptops.
Fortunately, our sponsors were able to send us our own Vive and laptop for development purposes.
\subsection{Documentation}
As this term's work mostly focused on documentation, we ran into several problems across the various documents.
The different structures of the documents was a reoccurring issue.
A lot of the requirements wern't clear and the IEEE format contradicted the suggested format from the assignment sheet and professors.


\section{Weekly Progress}
\subsection{Week 3}
This week we spent most of our time working on our Problem Statement which aimed to define the problem and make sure that everyone on the team and the sponsors were all on the same page.
In order to complete this assignment we had a conference call with our sponsors Mike Premi from Intel and Tim Delvin from Columbia to discuss their visions and scope of the project.
We were able to break down their requirements and discussed strategies going forward.

\subsection{Week 4}
We first spent some time working this week revising our Problem Statement.
Luckily our Problem Statement was used as an example during class so we received some very valuable feedback from Kirsten.
After that we moved on to getting the Unity Gaming Engine set up on our computers and familiarizing ourselves with the tool.

\subsection{Week 5}
This week we made progress in a few areas.
First we finished up our Problem Statement based on suggestions from Kirsten, the TA's, and our clients.
Next we made plans to get our own virtual reality gear from our clients.
Before this decision there had been discussion of us sharing VR gear with another Capstone team who is also doing a VR project with Intel.
While this would have been doable, we all decided it would be much simpler to recieve our own equipment.
We met with Erik Watterson from one of the other VR groups to learn about his project and to understand how to set up the HTC Vive.
We finished up the week by creating a rough draft of the requirements documentation for the project.

\subsection{Week 6}
This week we were able to meet with Christian from one of the other Intel teams.
We shared information and we plan to do some collaboration through the year.
He was able to bring down an HTC Vive, the stands, and a PC on Thursday for my team.
That evening my Team and I were able to set up the HTC Vive and test out a few free applications.
Towards the end of the week we were also able to complete the Software Requirements Specification for the project.

\subsection{Week 7}
This week we continued to experiment with VR. We've been playing some of the most popular titles from Steam Store.
Playing these polished games has given us a lot of ideas for features that we would like to implement in our game.
For example, we have seen many different ways to display menus, some good and bad.
Similarly, we have seen a couple different ways that developers deal with teleporting in game.
One of the most interesting applications we used was called Tilt Brush by Google.
This application does an amazing job of handling the user experience.
Tilt Brush offers an incredibly user friendly way to interact with a 3d paint brush.
It is intuitive and every single person on my team was able to understand exactly what to do within 10 seconds.
Moving forward we will strive to incorporate similar ideas into our project.
This week I also wrote 3 pieces for the Technology review.

\subsection{Week 8}
This week we made progress in a few areas. First we finished out Tech Review on Monday.
For us, this document ended up being very repetitive because most of our project will be completed in Unity and we were pretty much set on this tool before starting the document.
However, it was still a good exercise to explore how other gaming engines implement certain features.
This week we also completed a "Hello World" type program in Unity.
We were able to create a simple environment using basic assets from the Unity Asset Store and view that in the HTC Vive headset.
Fortunately there is a SteamVR plug-in in the Unity asset store which makes testing our application within the VR experience quick and easy.

\subsection{Week 9}
This week was relatively quiet as a group.
Most of my time was spent just doing some research and working on a basic prototype.
We continued to develop our prototype environment in Unity.
We learnt new aspects of the engine rapidly from online tutorials.
We also put together a statement about our fears and excitements for our project.
We briefly spoke with Mike Premi about our Technology Review and about getting NDA agreements for our team so we can collaborate more with Erik's team.

\subsection{Week 10}
This week we wrote our entire design document, which was super helpful for getting the details of our project into focus.
We plan to use this document as a guide throughout the development process.
Also, we finally all met with Erik Watterson and his team to share ideas and contact information.
We plan to collaborate with them more next term.

\section{Retrospective}
The following table represents a reflection of the previous development cycle.
It describes the positives, changes that need to be implimented, and actions required to implimennt these changes.

\begin{center}
\begin{tabular}{ |c || c | c | }
 \hline
 Positives & Deltas & Actions \\
 \hline \hline
 Collaboration with other & Need more development time & Schedule weekly team work sessions\\
 Intel VR team & & \\ \hline
 Good relationship with & More detail in project & -Learn from experience during \\ project sponsors & documentation & development \\
 & & -Submit documentation updates to \\
 & & professors and project sponsors\\ \hline
 Completed documentation & Need to obtain apparel & Setup meeting with Tim to discuss\\
 & assets from Columbia & \\ \hline
 Fully functional hardware setup & Decide on performace metrics & Further research on what\\
 & & makes VR experience successful\\ \hline
 Developed "Hello World" & NDA required to collaborate & Request NDA from Intel Sponsor \\
 Unity program & with other VR team & \\
 \hline
\end{tabular}
\end{center}
\section{Current Status}
As discussed above, the past term has primarily been devoted to developing documentation.
We spent the term carefully planning out every small aspect of our project.
While this was a lot of work, and not easy by any means, it will make the development process infinitely easier.
As we develop the project, the documentation is bound to change, but it will still serve as a blueprint for development.

Other than writing documentation, we also began learning Unity as none of our team had previous experience with the gaming engine.
At this point in time we have successfully completed a "Hello World" type program in Unity and displayed it on the Vive headset.
This was a big step for our team.
From here we have began the initial work on a functional prototype.
Currently it is simply a basic river and surrounding landscape that was constructed from some of the default Unity assets.
With the limited amount of free time we have found, this is a great start.
After the term ends our team will be staying in Corvallis for a couple of days to work on the prototype.
The goal is to have a basic functional VR fishing prototype by the beginning of next term.
This is a large undertaking, but our team is motivated to get flush out the prototype.
It will set us up for success going into next term.

%\clearpage
%\bibliographystyle{IEEEtran}
%\bibliography{designdoc}
\end{document}
