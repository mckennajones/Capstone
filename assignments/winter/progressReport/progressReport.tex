\documentclass[10pt,journal,compsoc,onecolumn, draftclsnofoot]{IEEEtran}

\usepackage{graphicx}
\usepackage{amssymb}
\usepackage{amsmath}
\usepackage{amsthm}
\usepackage{caption}

\usepackage{alltt}
\usepackage{float}
\usepackage{color}
\usepackage{url}

\usepackage{balance}
\usepackage[TABBOTCAP, tight]{subfigure}
\usepackage{enumitem}
\usepackage{pstricks, pst-node}

\usepackage{geometry}
\usepackage{pst-gantt}
\usepackage{tabu}

\geometry{textheight=8.5in, textwidth=6in}

%random comment

\newcommand{\cred}[1]{{\color{red}#1}}
\newcommand{\cblue}[1]{{\color{blue}#1}}

\graphicspath{ {diagrams/} }

\usepackage{hyperref}
\usepackage{geometry}
\usepackage{array}
\usepackage{titling}

\def\name{Jake Jeffreys, McKenna Jones, Spike Madden, Sean Marty}
\title{
EmbarkVR: Outdoor Virtual Reality Experience \\
CS Senior Capstone \\
Winter Midterm Progress Report\\
\vspace{1mm}
}
\author{Jake Jeffreys, McKenna Jones, Spike Madden, Sean Marty}
\date{12 Feburary 2016}

%pull in the necessary preamble matter for pygments output

%% The following metadata will show up in the PDF properties
\hypersetup{
  colorlinks = true,
  linkcolor = black,
  urlcolor = black,
  pdfauthor = {\name},
  pdfkeywords = {cs462 ``senior capstone''},
  pdftitle = {CS 462 Progress Report},
  pdfsubject = {CS 462 Progress Report},
  pdfpagemode = UseNone
}

\begin{document}
\begin{titlepage}
\maketitle
\vspace{1mm}
\begin{abstract}
This document describes the progress our group has made over the last term and where will will be going from here. It starts out with a brief overview of the project. It then moves in to discussing problems we encountered and how we able to solve them. Then there is a weekly progress report which breaks down our progress week by week. The document ends with a retrospective and some information related to our plans moving forward.
\end{abstract}
\vspace{1cm}
\end{titlepage}
\tableofcontents
\clearpage


\section{Project Overview}
The current typical retail experience is neither interactive nor immersive.
This project aims to create a functional and immersive virtual reality outdoor experience that promotes Columbia Sportswear gear to outdoor enthusiasts and newcomers alike.
Many outdoor activities initially require a large economic investment to get started.
This makes people less likely to try new outdoor activities.
The goal of the project is to develop an interactive product demonstration to combat this issue by providing customers with an experience of the outdoor activity before they purchase any gear related to it.
This will be accomplished through the use of the HTC Vive and the Unity Game Engine.

The final vision is to have this virtual reality experience placed in a Columbia Sportswear retail store.
Customers will be able to try on apparel in real life, and then perform movements, in the virtual reality experience, while wearing the gear.
This will give customers an idea of how the gear will feel on them when they are performing the same activity in real life.
The primary focus of this project is creating a fishing experience to promote the Performance Fishing Gear line at Columbia Sportswear.

\section{Current Status}
The following section gives insight into where there project currently stands, and the progress we have made since the previous progress report.

\subsection{Previous Work}
During winter break we spent a considerable amount of time becoming familiar with the Unity Game Engine and developing a basic prototype.
This prototype consisted of a basic terrain that would be suitable for our outdoor fishing experience.
This term our work has been a continuation of the progress we made on that prototype.

\subsection{Terrain}
The first area we have worked on is improving the terrain that we created in the prototype.
Originally we simply created an environment that closely resembled a location someone might fish at.
This term we have done a bit more research and roughly based our environment on a few popular fishing destinations around the Pacific NW.
Smith Rock State park in Central Oregon is one of these locations.
Our familiarity with the Unity terrain tools allowed us to do this.
It is fairly easy to use various terrain brushes to sculpt both mountainous and flatter areas.
One can also use the terrain tool to create bushes, trees, grass and other terrain related objects.

\subsection{User Interaction}
The other big area we have been working on is the user interaction aspect of our project, which will be our primary focus from here on out.
The Unity Package, NewtonVR has been the basis of user interaction thus far.
It makes it the process of creating realistic user interaction in a VR environment relatively painless.
You can essentially just add an \textit{InteractableItem} script to any GameObject, and without many adjustments, the user can interact with the item using the HTC Vive wands.
Instead of using the default SteamVR camera within Unity, we are now using the NVRCameraRig, to support this user interaction.
The NewtonVR package also makes it painless to replace the controllers with other objects in the experience.
For example, when the user picks up the fishing rod, the controller model will be replaced with the rod.
The final benefit of NewtonVR is that it makes that objects unable to pass through other objects.
This adds to the realism of the interaction.
Currently we have been working with nailing down the interaction with the fishing rod.
The hardest part of this interaction is making it feel realistic.
Currently we have a rigid fishing rod, and rigid fishing line, which clearly does not look as realistic as we would hope.
Unfortunately, none of us have much experience in this type of work.
We would ideally like there to be a realistic flex to both the rod and the fishing line.
One solution would be to use some type of 3d modeling program like Blender, but this might be out of the scope of this project.
Hopefully we will be able to find an asset on the Unity asset store to solve this issue.

\subsection{Project Focus}
Towards the beginning of this term we came up with the idea to have our VR experience be centered around a campsite.
We figured that this would be the perfect area of showcase some of the Columbia gear.
After talking with out client from Columbia he informed us that they already had a similar experience.
Therefore, we decided it might make more sense to have the experience be focused on user movement.
This way the user can be wearing Columbia gear in real life, while performing actions in VR, like fishing.
This changed our direction slightly, but not dramatically.
We will still keep the campsite, but it will no longer be the main attraction in our environment.

\subsection{Client Interaction}
At this point in the project we have been working with out client to get some essential assets for our project.
Firstly we need Columbia related assets.
This will include different types of apparel from the Performance Fishing Gear line at Columbia.
Ideally these apparel related assets will be ready to be placed on avatars, which our client has suggested is likely.
If this is the case, we hope to have a number of avatars, wearing Columbia gear, that will be walking around the campsite and river to showcase some of the Columbia gear.
We have also been discussing the need for a small budget to purchase some higher quality assets from the Unity Asset Store.
Up until now we have been relying on free assets, which are sufficient, but limit the realism of our experience.
We could focus some of our time on creating our own assets, but this time could be better spent elsewhere, like improving user interaction.

\section{Looking forward}
%what's left to do

\section{Problems Encountered (Jake Jeffreys)}
Throughout our project we have run into many different issues that have ranged in levels of severity.
Since this is the first time any of us have worked on a Virtual Reality project we knew there would be a lot we would need to overcome.
In this section we have laid out some of the largest issues we have run into so far.

\subsection{Collaboration with Other Virtual Reality Teams}
This year there are multiple senior design teams working on Virtual Reality projects.
During the first couple weeks after getting our project assignments we reached out to individuals on the other teams to start building relationships for future collaboration.
Erik Watterson and his team already had a Vive setup before we received our equipment so he was kind enough to share information related to getting started and ramping up in development.
We got a chance to try out his equipment and he shared some advice related to issues they had run into and overcome.
This ranged from software we would need to download to how to effectively setup the base stations.
This information helped us greatly when we received our gear and were getting started.
Moving forward we were excited about collaborating throughout the project but that’s when some roadblocks appeared.
It turns out that Erik’s group has all signed Nondisclosure Agreements which means they have to be incredibly careful with the information they share.
We attempted to get Nondisclosure agreements for our team so we could begin collaborating but we kept hitting dead ends.
Our two teams collaborated on a few of the reports during the first term but once development began there was very little we could share.
This proved to be quite frustrating because their team had already been working on their project for a few months before the beginning of school at their Intel internship.
This meant that they had more experience than us going into the first term which we could have learned from.
We also could have helped them solve some of there issues.
Instead we had to use online tutorials and just experiment with different techniques until we figured out solutions.

\subsection{Editor View}
None of us had any experience with Virtual Reality or Unity before starting this project but we were all excited to begin development.
The first step was understanding how to interact with Unity and navigate effectively.
There is a large community behind Unity which is super helpful but that doesn’t mean there aren’t some issues with documentation.
One of the first issues we ran into was the rendering distances.
during the development stage where we were placing foliage, the render distances were preventing us from seeing where we had already placed grass.
Fortunately, as we figure out how to better navigatew the editor view, it all become much easier.
This brings up our issue of moving around within the editor.
There was very limited information related to this so we had to figure it out for ourselves through experimentation.
We were able to move around in a relatively mediocre manner but we couldn’t figure out how to move laterally.
After looking at tutorials and on forums we came to the conclusion that it wasn’t a feature they had integrated into Unity yet.
A few weeks later we happened to be working on the project with a different mouse than before and we accidentally pressed down on the scroll wheel.
To our astonishment the whole screened move laterally.
This was incredibly exciting for us and improved our development efficiency exponentially.
We stopped using our mouse pads to do development and now always bring around wireless mice.
It’s unfortunate that this information is so hard to find for new users but we’re glad to have discovered this information relatively early on.

\subsection{Free Asset Availability}
One of the reasons we chose Unity over some of the other gaming engines was the community support behind it.
This makes it a great tool for new developers who have never created virtual reality games before.
It also means there are a wide range of free assets to use.
Towards the beginning of development this was great because we didn’t care as much about quality of assets as we did about building up the terrain and developing user interaction.
Now that our terrain is coming together we have started looking into adding realism to our environment.
This is where some issues have started to arise.
Realism is important to us in this project because we want it to feel as genuine and immersive as possible.
This is when free assets aren’t quite enough.
There are lots of free realism tools and techniques we have taken advantage of but when it comes to close user interaction, the objects need to be highly detailed.
In the world of Unity assets, detail costs money. We began making a list of assets we felt would dramatically improve the realism of our experience such as fishing equipment, fish, and high quality grass renderings.
These are the items that we felt would go the farthest in terms of value.
We then discussed this with our client and they were very open to our ideas.
We hope to get the approval shortly and move on to integrating these assets.

\subsection{Incorporating Columbia Gear}
From the beginning we were under the impression that our project was split between displaying Columbia gear and offering an interactive experience.
We spent a good amount of time brainstorming how to best display Columbia gear and allow user interaction of their clothing lines.
As we started development this term we have been having a hard time getting in contact with our clients to discuss our direction.
During week five we were able to talk with Tim from Columbia to get his input on what we've been working on.
At this point we discovered that Columbia has already worked on some projects that involve showcasing Columbia gear.
We were unaware of this and therefore we were trying to share our time evenly between the two goals.
We needed to change directions slightly and focus more on interactive activities.
Fortunately, we put our heads together and figured out a way to adapt our environment to follow this goal.
We had been working on a campsite to act as a base location and showcase Columbia gear.
Now we are going to keep it as a base and instead allow the user to walk around and interact with general objects.
This adds to immersion and creates a buffer between starting the experience and starting the fishing activity.


\section{Interesting Development Techniques}
%Maybe talk about some of the tools we are using in Unity??

\section{Images}
Below are some images that show the current state of our virtual reality experience.
\end{document}
