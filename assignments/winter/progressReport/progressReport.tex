\documentclass[10pt,journal,compsoc,onecolumn, draftclsnofoot]{IEEEtran}

\usepackage{graphicx}
\usepackage{amssymb}
\usepackage{amsmath}
\usepackage{amsthm}
\usepackage{caption}

\usepackage{alltt}
\usepackage{float}
\usepackage{color}
\usepackage{url}

\usepackage{balance}
\usepackage[TABBOTCAP, tight]{subfigure}
\usepackage{enumitem}
\usepackage{pstricks, pst-node}

\usepackage{geometry}
\usepackage{pst-gantt}
\usepackage{tabu}

\geometry{textheight=8.5in, textwidth=6in}

%random comment

\newcommand{\cred}[1]{{\color{red}#1}}
\newcommand{\cblue}[1]{{\color{blue}#1}}

\graphicspath{ {diagrams/} }

\usepackage{hyperref}
\usepackage{geometry}
\usepackage{array}
\usepackage{titling}

\def\name{Jake Jeffreys, McKenna Jones, Spike Madden, Sean Marty}
\title{
EmbarkVR: Outdoor Virtual Reality Experience \\
CS Senior Capstone \\
Winter Midterm Progress Report\\
\vspace{1mm}
}
\author{Jake Jeffreys, McKenna Jones, Spike Madden, Sean Marty}
\date{12 Feburary 2016}

%pull in the necessary preamble matter for pygments output

%% The following metadata will show up in the PDF properties
\hypersetup{
  colorlinks = true,
  linkcolor = black,
  urlcolor = black,
  pdfauthor = {\name},
  pdfkeywords = {cs462 ``senior capstone''},
  pdftitle = {CS 462 Progress Report},
  pdfsubject = {CS 462 Progress Report},
  pdfpagemode = UseNone
}

\begin{document}
\begin{titlepage}
\maketitle
\vspace{1mm}
\begin{abstract}
This document describes the progress our group has made over the last term and where will will be going from here. It starts out with a brief overview of the project. It then moves in to discussing problems we encountered and how we able to solve them. Then there is a weekly progress report which breaks down our progress week by week. The document ends with a retrospective and some information related to our plans moving forward.
\end{abstract}
\vspace{1cm}
\end{titlepage}
\tableofcontents
\clearpage

\section{Project Overview}
This project aims to create a functional and immersive virtual reality outdoor experience that promotes Columbia gear to outdoor enthusiasts and newcomers.
Many outdoor activities initially require a large mental and economic
investment to get started.
This makes people less likely to try new outdoor activities.
The goal of the project is to develop an interactive product demonstration to combat this issue.
This will be accomplished through the use of the HTC Vive and the Unity Game Engine.

\section{Current Status}
During winter break we spent a considerable amount of time becoming familiar with the Unity Game Engine and developing a basic prototype.
This prototype consisted of a basic terrain that would be suitable for our outdoor fishing experience.
This term our work has been a continuation of the progress we made on that prototype.

The first area we have worked on is improving the terrain that we created in the prototype.
Originally we simply created a environment that looked like someone might fish at.
This term we have done a bit more research and roughly based our environment on a few popular fishing destinations around the Pacific NW.
Our familiarity with the Unity terrain tools allowed us to do this.
It is fairly easy to use various terrain brushes to sculpt both mountainous and flatter areas.

The other big area we have been working on is the user interaction aspect of our project, which will be our primary focus from here on out.
The Unity Package, NewtonVR has been the basis of user interaction thus far.
It makes it the process of creating realistic user interaction in a VR environment relatively painless.
You can essentially just add an \textit{InteractableItem} script to any GameObject, and without many adjustments, the user can interact with them item using the HTC Vive wands.
Currently we have been working with nailing down the interaction with the fishing rod.
The hardest part of this interaction is making it feel realistic.
Currently we have a rigid fishing rod, and rigid fishing line, which clearly does not look as realistic as we would hope.
Unfortunately, none of us have much experience in this type of work.
We would ideally like there to be a realistic flex to the rod.
One solution would be to use some type of 3d modeling program like Blender, but this might be out of the scope of this project.

Towards the beginning of this term we came up with the idea to have our VR experience be centered around a campsite.
We figured that this would be the perfect area of showcase some of the Columbia gear.
After talking with out client from Columbia he informed us that it might make more sense to have the experience be focused on user movement.
This way the user can be wearing Columbia gear in real life, while performing actions in VR, like fishing.
This changed our direction slightly, but not dramatically.

\section{Looking forward}
%what's left to do

\section{Problems Encountered}

\section{Interesting Development Techniques}
%Maybe talk about some of the tools we are using in Unity??

\section{Images}
Below are some images that show the current state of our virtual reality experience.
\end{document}
