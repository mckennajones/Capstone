\documentclass[10pt,journal,compsoc,onecolumn, draftclsnofoot]{IEEEtran}

\usepackage{graphicx}
\usepackage{amssymb}
\usepackage{amsmath}
\usepackage{amsthm}
\usepackage{caption}

\usepackage{alltt}
\usepackage{float}
\usepackage{color}
\usepackage{url}

\usepackage{balance}
\usepackage[TABBOTCAP, tight]{subfigure}
\usepackage{enumitem}
\usepackage{pstricks, pst-node}

\usepackage{geometry}
\usepackage{pst-gantt}
\usepackage{tabu}

\geometry{textheight=8.5in, textwidth=6in}

%random comment

\newcommand{\cred}[1]{{\color{red}#1}}
\newcommand{\cblue}[1]{{\color{blue}#1}}

\graphicspath{ {diagrams/} }

\usepackage{hyperref}
\usepackage{geometry}
\usepackage{array}
\usepackage{titling}

\def\name{Jake Jeffreys, McKenna Jones, Spike Madden, Sean Marty}
\title{
EmbarkVR: Outdoor Virtual Reality Experience \\
CS Senior Capstone \\
Winter Midterm Progress Report\\
\vspace{1mm}
}
\author{Jake Jeffreys, McKenna Jones, Spike Madden, Sean Marty}
\date{12 Feburary 2016}

%pull in the necessary preamble matter for pygments output

%% The following metadata will show up in the PDF properties
\hypersetup{
  colorlinks = true,
  linkcolor = black,
  urlcolor = black,
  pdfauthor = {\name},
  pdfkeywords = {cs462 ``senior capstone''},
  pdftitle = {CS 462 Progress Report},
  pdfsubject = {CS 462 Progress Report},
  pdfpagemode = UseNone
}

\begin{document}
\begin{titlepage}
\maketitle
\vspace{1mm}
\begin{abstract}
This document describes the progress our group has made over the last term and where will will be going from here. It starts out with a brief overview of the project. It then moves in to discussing problems we encountered and how we able to solve them. Then there is a weekly progress report which breaks down our progress week by week. The document ends with a retrospective and some information related to our plans moving forward.
\end{abstract}
\vspace{1cm}
\end{titlepage}
\tableofcontents
\clearpage

\section{Project Overview}
This project aims to create a functional and immersive virtual reality outdoor experience that promotes Columbia gear to outdoor enthusiasts and newcomers.
Many outdoor activities initially require a large mental and economic
investment to get started.
This makes people less likely to try new outdoor activities.
The goal of the project is to develop an interactive product demonstration to combat this issue.
This will be accomplished through the use of the HTC Vive and the Unity Game Engine.

\section{Current Status}

\section{Looking forward}
%what's left to do

\section{Problems Encountered}

\section{Images}
Below are some images that show the current state of our virtual reality experience.
\end{document}
