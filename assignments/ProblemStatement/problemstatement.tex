\documentclass[letterpaper,10pt,titlepage]{article}

\usepackage{graphicx}
\usepackage{amssymb}
\usepackage{amsmath}
\usepackage{amsthm}

\usepackage{alltt}
\usepackage{float}
\usepackage{color}
\usepackage{url}

\usepackage{balance}
\usepackage[TABBOTCAP, tight]{subfigure}
\usepackage{enumitem}
\usepackage{pstricks, pst-node}

\usepackage{geometry}
\geometry{textheight=8.5in, textwidth=6in}

%random comment

\newcommand{\cred}[1]{{\color{red}#1}}
\newcommand{\cblue}[1]{{\color{blue}#1}}

\usepackage{hyperref}
\usepackage{geometry}

\def\name{Jake Jeffreys, McKenna Jones, Spike Madden, Sean Marty}
\usepackage{titling}
\title{CS461: Problem Statement}
\author{Jake Jeffreys, McKenna Jones, Spike Madden, Sean Marty}
\date{October 10, 2016}

%pull in the necessary preamble matter for pygments output

%% The following metadata will show up in the PDF properties
\hypersetup{
  colorlinks = true,
  urlcolor = black,
  pdfauthor = {\name},
  pdfkeywords = {cs461 ``senior capstone''},
  pdftitle = {CS 461 Problem Statement},
  pdfsubject = {CS 461 Problem Statement},
  pdfpagemode = UseNone
}

\begin{document}
\begin{titlepage}
\maketitle
\begin{abstract}
Many outdoor activities these days initially require a large mental and economic
investment to get started. This makes people less likely to try new outdoor
activities. The goal of the project is to develop an interactive product
demonstration with virtual reality to combat this issue. This project has the
potential to inspire people to get outdoors and try new things by first getting
them comfortable in new environments or performing new movements. It strives to
make outdoor activities accessible to everyone no matter their experience level.
This project will not only inspire but also improve the retail experience by
making it more immersive, interactive, and informative while being
entertaining. The main tool being used is Unity Gaming Engine in tandem with an
HTC Vive Virtual Reality System. The objective of this project is to create a
functional VR outdoor experience ready to be used in a Columbia retail store by
May of 2017.
\end{abstract}
\end{titlepage}

\section{Problem Definition}
This project aims to create a virtual reality application that allows for
outdoor enthusiasts to interact in immersive and responsive environments with
outdoor sports apparel and equipment. The retail experience needs to be
informative yet entertaining to ensure that the product inspires consumers to
get outdoors.

\section{Proposed Solution}
The final solution to this problem will be in the form of an interactive
product that makes use of virtual reality. Ideally we would like to create an
experience in a retail store that is both immersive and interactive. What this
will most likely look like is a room and/or section of the retail store when
customers can go to use our virtual reality product.

While using the final product, the consumer will be able test out different
outdoor gear products in the type of environments that they would be used in
the real world. This allows consumers to gain exposure to the type of
environments where the products would be used, before they decide to make a
purchace. This has the possibility of people very useful for activities that
require a large initial investment. Fishing is one example of this. Consumers
would be able to experience fishing virtually, before deciding if they would
actually like to invest their time and money into the sport.

At the Expo we hope to similate a similar environment as the one which will be
presented in the retail environment. This would give students, professors and
other attendees of the Expo the best idea of what our project is all about.

\end{document}
