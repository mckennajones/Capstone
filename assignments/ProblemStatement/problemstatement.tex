%\documentclass[letterpaper,10pt,titlepage]{IEEEtran}
\documentclass[10pt, oneside,onecolumn,draftclsnofoot]{IEEEtran}

\usepackage{graphicx}
\usepackage{amssymb}
\usepackage{amsmath}
\usepackage{amsthm}

\usepackage{alltt}
\usepackage{float}
\usepackage{color}
\usepackage{url}

\usepackage{balance}
\usepackage[TABBOTCAP, tight]{subfigure}
\usepackage{enumitem}
\usepackage{pstricks, pst-node}

\usepackage{geometry}
\geometry{textheight=8.5in, textwidth=6in}

%random comment

\newcommand{\cred}[1]{{\color{red}#1}}
\newcommand{\cblue}[1]{{\color{blue}#1}}

\usepackage{hyperref}
\usepackage{geometry}

\def\name{Jake Jeffreys, McKenna Jones, Spike Madden, Sean Marty}
\usepackage{titling}
\title{CS461: Problem Statement}
\author{Jake Jeffreys, McKenna Jones, Spike Madden, Sean Marty}
\date{October 10, 2016}

%pull in the necessary preamble matter for pygments output

%% The following metadata will show up in the PDF properties
\hypersetup{
  colorlinks = true,
  urlcolor = black,
  pdfauthor = {\name},
  pdfkeywords = {cs461 ``senior capstone''},
  pdftitle = {CS 461 Problem Statement},
  pdfsubject = {CS 461 Problem Statement},
  pdfpagemode = UseNone
}

\begin{document}
\begin{titlepage}
\maketitle
\vspace{3cm}
\begin{abstract}
Many outdoor activities these days initially require a large mental and economic
investment to get started. This makes people less likely to try new outdoor
activities. The goal of the project is to develop an interactive product
demonstration with virtual reality to combat this issue. This project has the
potential to inspire people to get outdoors and try new things by first getting
them comfortable in new environments or performing new movements. It strives to
make outdoor activities accessible to everyone no matter their experience level.
This project will not only inspire but also improve the retail experience by
making it more immersive, interactive, and informative while being
entertaining. The main tool being used is Unity Gaming Engine in tandem with an
HTC Vive Virtual Reality System. The objective of this project is to create a
functional VR outdoor experience ready to be piloted in a Columbia retail store by
May of 2017.
\end{abstract}
\end{titlepage}

\section{Problem Definition}
This project aims to create a virtual reality application that allows for
outdoor enthusiasts to interact in immersive and responsive environments with
outdoor sports apparel and equipment. The retail experience needs to be
informative yet entertaining to ensure that the product inspires consumers to
get outdoors.


\section{Proposed Solution}

The final solution to this problem will be in the form of an interactive
product that makes use of virtual reality. Ideally we would like to create an
experience in a retail store that is both immersive and interactive. What this
will most likely look like is a room or section of a retail store where
customers can go to use our virtual reality project. The three main goals of
this project are to create brand engagement, reach customers who might have not
been reached otherwise, and hopefully create sales for Columbia Sportswear.

While using the final product, the consumer will be able to test out different
Columbia products in the type of environments that they would be used in
the real world. The goal is to create an experience that appeals to all senses.
Along with the virtual reality portion of the experience there may also be
smells, sounds, and things to touch. This allows consumers to gain exposure to
the type of environments where the products would be used before they decide to
make a purchace. It also allows the customer to experience how the product will
feel in that particular environment. This has the possibility of being very
useful for activities that require a large initial investment. Fishing is one
example of this. Consumers would be able to experience fishing with Columbia
Sportswear gear, before deciding if they would actually like to invest their
time and money into the sport. This would allow Columbia Sportwear to reach
customers who may have not otherwise had an interaction with their brand.

Social media is currently an invaluable resource when it comes to marketing.
We hope to leverage this in our project. By adding the ability to share
virtual reality experiences that users are having in retail stores to social
media sites, Columbia Sportswear will be able to reach even more customers.
This may be in the form of sharing either short 360 videos or photos on sites
like Facebook and Youtube, which support this functionality. Not only will this
create brand awareness, but it will also create awareness of our project.

At the Expo in the Spring the hope is to mimic the retail environment in which
the project will ultimately be used. At our booth will be various pieces of
Columbia gear for users to test. If the full VR experience cannot be set up
at Expo, we will provide video and other demonstrations of our work. This would
give students, professors and other attendees of the Expo the best idea of
what the project is all about.

\section{Performance Metrics}
The first thing that can be done to measure the success of this project is
monitor customer experiences during the user testing phase. Among many other
things, the application strives to inspire and entertain which should be visible
among customers when interacting with the application. During the user testing
phase there will also be an opportunity to receive oral or written feedback
before and after the experience from users to determine how it has impacted the
likelihood that they try the activity outdoors or purchase related clothing and
equipment. There will also be a way for users to share their virtual reality
experience online which can be used to test whether or not users enjoyed it
enough to share it. Another aim of this project is to convert sales for Columbia
so within the application we will include a tool that allows customers to flag
products. Monitoring the activity of this feature will give insight into the
ability for the experience to impact user purchases.

\vspace{3cm}

\noindent\begin{tabular}{ll}
\makebox[2.5in]{\hrulefill} & \makebox[2.5in]{\hrulefill}\\
Intel Sponsor & Date\\[8ex]% adds space between the two sets of signatures
\makebox[2.5in]{\hrulefill} & \makebox[2.5in]{\hrulefill}\\
Columbia Sponsor & Date\\[8ex]% adds space between the two sets of signatures
\makebox[2.5in]{\hrulefill} & \makebox[2.5in]{\hrulefill}\\[2ex]
\makebox[2.5in]{\hrulefill} & \makebox[2.5in]{\hrulefill}\\[2ex]
\makebox[2.5in]{\hrulefill} & \makebox[2.5in]{\hrulefill}\\[2ex]
\makebox[2.5in]{\hrulefill} & \makebox[2.5in]{\hrulefill}\\
Student Team Members & Date\\
\end{tabular}




\end{document}
