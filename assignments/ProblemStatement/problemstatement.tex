\documentclass[letterpaper,10pt,titlepage]{article}

\usepackage{graphicx}
\usepackage{amssymb}
\usepackage{amsmath}
\usepackage{amsthm}

\usepackage{alltt}
\usepackage{float}
\usepackage{color}
\usepackage{url}

\usepackage{balance}
\usepackage[TABBOTCAP, tight]{subfigure}
\usepackage{enumitem}
\usepackage{pstricks, pst-node}

\usepackage{geometry}
\geometry{textheight=8.5in, textwidth=6in}

%random comment

\newcommand{\cred}[1]{{\color{red}#1}}
\newcommand{\cblue}[1]{{\color{blue}#1}}

\usepackage{hyperref}
\usepackage{geometry}

\def\name{Jake Jeffreys, McKenna Jones, Spike Madden, Sean Marty}

%pull in the necessary preamble matter for pygments output

%% The following metadata will show up in the PDF properties
\hypersetup{
  colorlinks = true,
  urlcolor = black,
  pdfauthor = {\name},
  pdfkeywords = {cs461 ``senior capstone''},
  pdftitle = {CS 461 - Senior Capstone Project},
  pdfsubject = {CS 461 Capstone},
  pdfpagemode = UseNone
}

\begin{document}

Many outdoor activities these days initially require a large mental and economic
investment to get started. This makes people less likely to try new outdoor
activities. The goal of the project is to develop an interactive product
demonstration with virtual reality to combat this issue. This project has the
potential to inspire people to get outdoors and try new things by first getting
them comfortable in new environments or performing new movements. It strives to
make outdoor activities accessible to everyone no matter their experience level.
This project will not only inspire but also improve the retail experience by
making it more immersive, interactive, and informative while being
entertaining. The main tool being used is Unity Gaming Engine in tandem with an
HTC Vive Virtual Reality System. The objective of this project is to create a
functional VR outdoor experience ready to be used in a Columbia retail store by
May of 2017.

\end{document}
